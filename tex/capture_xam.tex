% $Id:$
%  --------------------------------------------------------------------------
% cppad_mixed: C++ Laplace Approximation of Mixed Effects Models
%           Copyright (C) 2014-16 University of Washington
%              (Bradley M. Bell bradbell@uw.edu)
%
% This program is distributed under the terms of the
%	     GNU Affero General Public License version 3.0 or later
% see http://www.gnu.org/licenses/agpl.txt
% ---------------------------------------------------------------------------
% Beamer template file.

\documentclass{beamer}

% Choose theme; e.g., default, Wasaw
\usetheme{default}

% To avoid font warnings caused by Beamer
\usepackage{lmodern}

% For web links
\usepackage{hyperref}

% use lable instead of default icon in bibliography
\setbeamertemplate{bibliography item}{\insertbiblabel}

% Binary file not included for this talk
% \pgfdeclareimage[height=0.5cm]{ihme-apl-logo}{ihme-apl-logo}
% \logo{\pgfuseimage{ihme-apl-logo}}

% Common macros
\newcommand{\Green}[1]{\textcolor{green}{#1}}
\newcommand{\Blue}[1]{\textcolor{blue}{#1}}
\newcommand{\B}[1]{{\bf #1}}
\newcommand{\R}[1]{{\rm #1}}
\newcommand{\Section}[1]{
	\section{#1}
	\begin{frame}
	\begin{center}
	\Blue{ \Large{#1} }
	\end{center}
	\end{frame}
}
%
\title[cppad\_mixed]{
The cppad\_mixed Capture Example and Speed Test}
\author{Bradley M. Bell}
\institute{
	Applied Physics Laboratory, \\
	Health Metrics and Evaluation, \\
	University of Washington, \\
	{\tt bradbell@uw.edu}
}
\date[2016-06-(20-24)]
{AMDMB and TMB Developer's Workshop \\ June, 2016}
%
% This example definition is intended to be replaced by a renewcommand that
% defines a specfic \TabbedItem just before it is used.
\newcommand{\TabbedItem}[1]{
\item
\begin{tabbing}
12345\=67890\=12345\=67890\=12345\=\kill
#1
\end{tabbing}
}

\begin{document}

\begin{frame}
	\titlepage
\end{frame}

\begin{frame}{Contents}
\tableofcontents[pausesections]
\end{frame}


% ===========================================================================
\Section{Introduction}
% --------------------------------------------------------------------------

\begin{frame}{cppad\_mixed}
C++ Laplace approximation for non-linear mixed effect models
\cite{BellCppADMixed}.
\pause

\begin{itemize}

\item
Open souce with souce code control on \texttt{github}.
\pause

\item
Example of CppAD addon; e.g., traps CppAD error messages.
\pause

\item
Easy to used debugger and asserts in complex modeling situations.
\pause

\item
Can be used as a front end for TMB like applications in any language; e.g,
python.
\pause

\item
Can be used for starting point and/or proposal sampling in MCMC method.
\pause

\item
Gaussian or Laplace priors and non-linear constraints
on fixed effects.
\pause

\end{itemize}
\end{frame}

% --------------------------------------------------------------------------

\begin{frame}{captrue\_xam}
This is a \texttt{cppad\_mixed} example; see  \cite{Royle2004}.

\begin{itemize}

\item
Tests both optimization and posterior sampling.
\pause

\item
Reports timing test results.
\pause

\item
Reports correctness test results.
\pause

\item
Reports memory usage results.
\pause

\item
Separate control of data set size.
\pause

\item
Separate control of number of random effects.
\pause

\item
Separate control of computational load.
\pause

\item
Compare Newton and quasi-Newton methods.
\pause

\item
Interesting numerical AD method used to avoid overflow.
\pause

\item
Optional constraints.

\end{itemize}
\end{frame}

% ===========================================================================
\Section{Data Model}
% --------------------------------------------------------------------------

\begin{frame}{Data Given Population Size and Capture Probability}
\begin{itemize}

\item
$y_{i,t}$ is the number of captures at location $i$ and time $t$.
\pause

\item
$N_i$ is the size of the population at location $i$.
\pause

\item
$q_t$ is the probability of capture at time $t$.
\pause

\item
The conditional probability of $y_{i,t}$ given $N_i$ and $q_t$ is
\[
\B{p} ( y_{i,t} | N_i , q_t )
=
\left( \begin{array}{c} N_i \\ y_{i,t} \end{array} \right)
q_t^{y(i,t)} \left( 1 - q_t\right)^{y(i,t)}
\]

\end{itemize}
\end{frame}
% --------------------------------------------------------------------------

\begin{frame}{Population Size Given Fixed Effects}
\begin{itemize}

\item
$\theta_0$ is the mean of $N_i$ .
\pause

\item
$\theta$ is the vector of fixed effects.
\pause

\item
We use a Poisson distribution to model the probability of $N_i$
given $\theta$
\[
\B{p} ( N_i | \theta  )
=
\theta_0^{N(i)} \frac{ \exp[ - \theta_0 ] }{ N_i ! }
\]

\end{itemize}
\end{frame}
% --------------------------------------------------------------------------

\begin{frame}{Capture Probability Given Fixed and Random Effects}
\begin{itemize}

\item
$\theta_1$ is the mean of the logit of $q_t$.
\pause

\item
$u_t$ is the random effect at time $t$.
\pause

\item
$u$ is the vector of random effects.

\item
The logit of $q_t$ is
\[
	\log [ q_t / ( 1 - q_t ) ] = u_t + \theta_1
\]
\pause

\item
The probability of capture at time $t$ given $\theta$ and $u$ is
\[
	q_t ( \theta , u )
	=
	\B{p} ( q_t | \theta , u )
	=
	[ 1 + \exp( - u_t - \theta_1 ) ]^{-1}
\]


\end{itemize}
\end{frame}
% --------------------------------------------------------------------------

\begin{frame}{Data Given Fixed and Random Effects}
\begin{itemize}

\item
$y_i$ vector of data at location $i$ and all times.
\pause

\item
$T$ is the number of times.
\pause

\item
The probability of $y_i$ given $N_i$, $\theta$, and $u$ is
\[
\B{p}( y_i | N_i, \theta , u )
=
\prod_{t=0}^{T-1}
\left( \begin{array}{c} {N(i)} \\ y_{i,t} \end{array} \right)
	q_t ( \theta , u)^{y(i,t)}
	\left( 1 - q_t( \theta , u) \right)^{y(i,t)}
\]
\pause

\item
$K$ is maximum population in truncation of infinite summation.
\pause

\item
Probability of $y_i$ given $\theta$, $u$, is modeled using the sum
\[
\B{p}( y_i | \theta , u )
=
\sum_{k=0}^K \B{p}( y_i | N_i=k, \theta , u ) \B{p}( N_i=k | \theta )
\]
\pause

\end{itemize}
\end{frame}
% --------------------------------------------------------------------------
%
\begin{frame}{Data Given Fixed and Random Effects}
\begin{itemize}

\item
$y$ vector of data at all locations and all times.
\pause
%
\item
$R$ is the number of locations.
\pause
%
\item
The probability of $y$ given $\theta$, and $u$ is
\[
\B{p}( y | \theta , u )
=
\prod_{i=0}^{R-1} \B{p}( y_i | \theta , u )
=
\]
\pause
\[
\prod_{i=0}^{R-1}
\left[
\sum_{k=0}^{K-1}
\theta_0^k \frac{ \exp[ - \theta_0 ] }{ k ! }
\prod_{t=0}^{T-1}
\left( \begin{array}{c} {k} \\ y_{i,t} \end{array} \right)
	q_t ( \theta , u)^{y(i,t)}
	\left( 1 - q_t( \theta , u) \right)^{y(i,t)}
\right]
\]

\end{itemize}
\end{frame}
% ===========================================================================
\Section{Program Input}
% --------------------------------------------------------------------------

\begin{frame}{Random Likelihood Function}
\begin{itemize}

\item
$\theta_2$ is the standard deviation of the logit of $q_t$.
\pause

\item
We use a normal distribution for $u$ given $\theta$,
\[
\B{p} ( u | \theta )
=
\prod_{t=0}^{T-1}
	\frac{1}{ \theta_2 \sqrt{ 2 \pi } }
		\exp \left[ - \frac{1}{2} \frac{ u_t^2 }{ \theta_2^2 } \right]
\]
\pause

\item
The user defined random likelihood for this example is
\[
f( \theta , u )
=
- \log [ \B{p} ( y | \theta , u) \B{p} ( u | \theta ) ]
\]

\end{itemize}
\end{frame}
% --------------------------------------------------------------------------
\begin{frame}{Other User Defined Functions}
\begin{itemize}

\item
The is no prior for the fixed effects; i.e.,
$\B{p}( \theta )$.
\pause

\item
The is no data that only depends on fixed effects; i.e.,
$\B{p}(z| \theta )$.
\pause

\item
The is no fixed constraint functions; i.e,
$c( \theta )$.
\pause

\item
\[
\hat{u} ( \theta ) = \R{argmin} \; f( \theta , u ) \; \R{w.r.t.} \; u
\]
\pause

\item
$A$ is the random constraint matrix.

\item
The random constraint function is
\[
0
=
A \hat{u} ( \theta )
=
\hat{u}_1 ( \theta ) + \cdots + \hat{u}_{T-1} ( \theta )
\]

\end{itemize}
\end{frame}
% --------------------------------------------------------------------------
\begin{frame}{Command Line Arguments}
\fontsize{10}{11}\selectfont
\begin{itemize}

\item
random\_seed: \hspace{4em} a non-negative integer.
\pause

\item
number\_fixed\_samples: \hspace{0.2em} a positive integar.
\pause

\item
number\_locations: \hspace{2em} a positive integer , $R$.
\pause

\item
max\_population: \hspace{2.7em} a positive integer , $K$.
\pause

\item
mean\_population: \hspace{2.2em} a positive real , $\theta_0$.
\pause

\item
mean\_logit\_probability: \hspace{0em} a real , $\theta_1$.
\pause

\item
std\_logit\_probability: \hspace{1em} a positive real , $\theta_2$.
\pause

\item
quasi\_fixed: \hspace{4.8em} true or false.
\pause

\item
random\_constraint: \hspace{1.6em} true or false.
\pause

\item
trace\_optimize\_fixed: \hspace{1em} true or false.
\pause

\end{itemize}
\end{frame}

% --------------------------------------------------------------------------
\begin{frame}{Memory and Timing Output}
\begin{itemize}

\item
initialization\_bytes
\pause

\item
initialization\_seconds
\pause

\item
optimize\_fixed\_seconds
\pause

\item
optimize\_random\_seconds
\pause

\item
information\_mat\_seconds
\pause

\item
sample\_fixed\_seconds
\pause

\end{itemize}
\end{frame}

% --------------------------------------------------------------------------
\begin{frame}{Estimate Output}
\begin{itemize}

\item
sum\_random\_effects
\pause

\item
mean\_population\_estimate
\pause

\item
mean\_logit\_probability\_estimate
\pause

\item
std\_logit\_probability\_estimate
\pause

\item
mean\_population\_std
\pause

\item
mean\_logit\_probability\_std
\pause

\item
std\_logit\_probability\_std
\pause

\item
mean\_population\_ratio
\pause

\item
mean\_logit\_probability\_ratio
\pause

\item
std\_logit\_probability\_ratio
\pause


\end{itemize}
\end{frame}

% ===========================================================================
\Section{Bibliography}
%
\bibliography{capture_xam}{}
\bibliographystyle{alpha}


\end{document}
